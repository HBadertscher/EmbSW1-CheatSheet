\newpage
\section{Delegates}
\subsection{Normale Delegates}
Delegate erstellen:
\begin{lstlisting}[style=Csharp]
	 delegate void LogWriter(string message);
\end{lstlisting}

Klasse FileLogger 
\begin{lstlisting}[style=Csharp]
    public class FileLogger
    {
        private StreamWriter w;

        public FileLogger(string path)
        {
            w = File.CreateText(path);
        }

        public void Print(string msg)
        {
            w.WriteLine(msg);
        }

        public void Flush()
        {
            w.Flush();
        }

        public void Close()
        {
            w.Close();
        }
    }
\end{lstlisting}

Klasse Test:
\begin{lstlisting}[style=Csharp]
	public class Test
	{

	    // Klassenvariable vom Typ "LogWriter" erstellen (Name Log)
	    private static LogWriter Log;
	
	    // Methode "Print" fuer Ausgabe auf dem Bildschirm erstellen
	    private static void Print(string s)
	    {
	        Console.WriteLine(s);
	    }
	
	    public static void Main(string[] arg)
	    {
	        // Logger Instanz erstellen
	        FileLogger logger = new FileLogger("U:/log.txt");
	
	        // Event fuer Ausgabe auf dem Bildschirm anmelden
	        Log += Print;
	        // Event fuer Ausgabe in File anmelden
	        Log += logger.Print; 
	
	        // Logeintraege erstellen
	        Log("Log. Loglog.");
	
	        // Buffer der "Logger" Instanz schreiben und danach schliessen
	        logger.Flush();
	        logger.Close();
	    }
 \end{lstlisting}
 
 \newpage
 \subsection{Events}
 
 Definition eines normalen Delegate:
 \begin{lstlisting}[style=Csharp]
 	public delegate void CountChangedHandler(CountChangedEventArgs e);
 \end{lstlisting}
 
 Definition einer Callback Arguments Klasse:
 \begin{lstlisting}[style=Csharp]
  public class CountChangedEventArgs : EventArgs
  {
      private int mOldValue;
      private int mNewValue;
 
      public CountChangedEventArgs(int oldValue, int newValue)
      {
          mOldValue = oldValue;
          mNewValue = newValue;
      }
 
      public int OldValue
      {
          get { return mOldValue; }
      }
 
      public int NewValue
      {
          get { return mNewValue; }
      }
  }
 \end{lstlisting}
 
 Definition des Counters
 \begin{lstlisting}[style=Csharp]
  class Counter
  {
      private int count;
      
      // Event vom Typ "CountChangedHandler" erstellen
      public event CountChangedHandler countChange;
      
      public int Count
      {
          set
          {
              CountChangedEventArgs e = new CountChangedEventArgs(count, value);
              countChange(e);
              count = value;
          }
          get { return count; }
      }
 
      public void Increment()
      {
          Count++;
      }
 
      public void Decrement()
      {
          Count--;
      }
 
      public void Reset()
      {
          Count = 0;
      }
  }
 \end{lstlisting}
 
 Hauptprogramm:
 \begin{lstlisting}[style=Csharp]
  class Program
  {
 
      public static void CountChangeListener(CountChangedEventArgs e)
      {
          Console.WriteLine("Counter Value changed from {0} to {1}", e.OldValue, e.NewValue);
      }
 
      static void Main(string[] args)
      {
 
          Counter myCount = new Counter();
          myCount.countChange += CountChangeListener;
 
          myCount.Count = 5;
 
          myCount.Increment();
          myCount.Decrement();
 
          myCount.Reset();
      }
  }
 \end{lstlisting}