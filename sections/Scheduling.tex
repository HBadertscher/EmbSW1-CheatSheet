\section{Scheduling}

\subsection{Auslastung / Utilization}
\begin{multicols}{2}

	$u_i = \frac{e_i}{p_i}$ \\
	
	mit \\
	$u_i$: Ausführungsfaktor des Task $i$ \\
	$e_i$: Executionzeit des Task $i$ \\
	$p_i$: Periodenzeit des Task $i$ \\

	
	Gesamtauslastung $U$ für $n$ periodische Tasks:
	\begin{equation*}
		U = \sum\limits_{i=1}^{n} u_i = \sum\limits_{i=1}^{n} \frac{e_i}{p_i}
	\end{equation*}

	\adjustbox{width=0.9\linewidth}{
	\begin{tabular}{|l|l|l|} \hline
		\textbf{Utilization (\%)} & \textbf{Zone Type} & \textbf{Typical Application} \\ \hline
		0-25 & excess processing power & various \\
		26-50 & very safe & various \\
		51-68 & safe & various \\
		69 & theoretical limit & embedded systems \\
		70-82 & questionable & embedded systems \\
		83-99 & dangerous & embedded systems \\
		100+ & overload & stressed systems \\ \hline
	\end{tabular}}


\columnbreak

\subsection{Rate monotonic scheduling}

Systemauslastung $U$ bei $n$ periodischen Tasks muss kleiner sein als: \\

\begin{equation*}
	U \leq n \cdot \left( 2^{\frac{1}{n}} - 1\right)
\end{equation*}

\begin{tabularx}{0.8\linewidth}{|p{1cm}|X||p{1cm}|X|} \hline
	$\mathbf{n}$ & $\mathbf{U [ \% ]}$ & $\mathbf{n}$ & $\mathbf{U [ \% ]}$ \\ \hline \hline
	$2$ & $82.4$ & $5$ & $74.4$ \\
	$3$ & $78.0$ & $10$ & $71.1$ \\
	$4$ & $75.7$ & $\infty$ & $\ln 2 \approx 69.3$ \\ \hline
\end{tabularx}

\end{multicols}